\documentclass[12pt]{article}

\usepackage{mlmodern}
\usepackage[T1]{fontenc}

\usepackage{ulem}
\usepackage{float}
\usepackage{graphicx}
\usepackage{fancyhdr}
\usepackage{multirow}
\usepackage{setspace}
\usepackage[spanish]{babel}
\usepackage[left=1.5cm, right=1.5cm, top=4cm, bottom=2.5cm, headheight=2cm, headsep=1cm]{geometry}

\pagestyle{fancy}
\fancyhead[C]
{
    \makebox[\textwidth]{
        \hspace{-0.2\textwidth}
        \begin{minipage}[c][2.5cm]{0.2\textwidth}
            \includegraphics[width=2.5cm]{logo.value}
        \end{minipage}
        \hspace{0.01\textwidth}
        \begin{minipage}[c][2.5cm]{0.6\textwidth}
            \centering
            \textbf{\LARGE Colegio Bautista Libertad}\\[0.2cm]
            \textbf{\Large Historial académico}\\[0.1cm]
            Contacto: info@cbl-edu.com
        \end{minipage}
    }
}

\setlength{\parskip}{1.5mm}
\setlength{\parindent}{0pt}
\renewcommand{\headrulewidth}{0pt}

\pagenumbering{gobble}

\newcommand{\aLine}[1]{\makebox[\linewidth]{#1}}

\newcommand{\aDupleLeft}[2]{\textbf{#1}\hspace{2mm} #2 \hfill}
\newcommand{\aDupleCenter}[2]{\hfill \aDupleLeft{#1}{#2}}
\newcommand{\aDupleRight}[2]{\hfill\textbf{#1}\hspace{2mm} #2}

\newcommand{\enrollmentName}{enrollment.value}
\newcommand{\minedId}{mined.id.value}
\newcommand{\studentName}{student.name.value}
\newcommand{\schoolYear}{schoolyear.value}

\newcommand{\firstAverage}{first.average.value}
\newcommand{\secondAverage}{second.average.value}
\newcommand{\thirdAverage}{third.average.value}
\newcommand{\fourthAverage}{fourth.average.value}
\newcommand{\finalAverage}{final.average.value}

\newcommand{\userAlias}{user.alias.value}
\newcommand{\currentDate}{current.date.value}
\newcommand{\currentDatetime}{current.datetime.value}
\makeatletter
\renewcommand\@makefntext[1]{\noindent #1}
\makeatother

\begin{document}
    \hfill\textbf{Año: }\schoolYear\\
    
    El estudiante \textbf{\studentName} con código mined \textbf{\minedId} quien cursa (o cursó) \textbf{\enrollmentName}
    del plan de estudio de educación secundaria para el año lectivo \textbf{\schoolYear} del turno diurno, obtuvo las calificaciones siguientes.

    \begin{table}[h]
        \centering
        \begin{tabular}{|l||*{4}{c|}|*{4}{c|}|*{2}{c|}}
            \hline
            
            & \multicolumn{4}{c||}{I Semestre} &
            \multicolumn{4}{c||}{II Semestre} &
            \multicolumn{2}{c|}{Calificación}\\\cline{2-9}
            
            \multirow{3}{*}{\large\hspace{0.6cm} Áreas/Disciplinas} &
            \multicolumn{2}{c|}{I Corte} &
            \multicolumn{2}{c||}{II Corte} &
            \multicolumn{2}{c|}{III Corte} &
            \multicolumn{2}{c||}{IV Corte} &
            \multicolumn{2}{c|}{final}\\\cline{2-11}
            
            & \rotatebox{90}{\tiny Cualitativa} & \rotatebox{90}{\tiny Cuantitativa } &
            \rotatebox{90}{\tiny Cualitativa} & \rotatebox{90}{\tiny Cuantitativa } &
            \rotatebox{90}{\tiny Cualitativa} & \rotatebox{90}{\tiny Cuantitativa } &
            \rotatebox{90}{\tiny Cualitativa} & \rotatebox{90}{\tiny Cuantitativa } &
            \rotatebox{90}{\tiny Cualitativa} & \rotatebox{90}{\tiny Cuantitativa } \\\hline
            
            detail.value
            
            {\footnotesize Promedio} &
            \multicolumn{2}{r|}{\firstAverage} &
            \multicolumn{2}{r||}{\secondAverage} &
            \multicolumn{2}{r|}{\thirdAverage} &
            \multicolumn{2}{r||}{\fourthAverage} &
            \multicolumn{2}{r|}{\finalAverage} \\\hline
        \end{tabular}
    \end{table}

    Escala de calificaciones en el rango de 0 a 100 puntos, escala utilizada desde 2009 hasta la fecha.
    La nota mínima para aprobar es \textbf{60 puntos.}
    \begin{table}[h]
        \centering
        \begin{tabular}{l | l | l}
            \hline
            Nivel de competencias & Cualitativo & Cuantitativo\\\hline
            Aprendizaje avanzado & AA & 90 - 100\\
            Aprendizaje satisfactorio & AS & 76 - 89\\
            Aprendizaje elemental & AE & 60 - 75\\
            Aprendizaje inicial & AI & 00 - 59
        \end{tabular}
    \end{table}

    Documento emitido en la cuidad de Managua, el \currentDate.
    
    \footnotetext{Impreso por wsmcbl el \currentDatetime, \userAlias.}
\end{document}